\documentclass[a4paper,12pt]{article}

\usepackage[hidelinks]{hyperref}
\usepackage{amsmath}
\usepackage{mathtools}
\usepackage{shorttoc}
\usepackage{cmap}
\usepackage[T2A]{fontenc}
\usepackage[utf8]{inputenc}
\usepackage[english, russian]{babel}
\usepackage{xcolor}
\usepackage{graphicx}
\usepackage{float}
\graphicspath{{./images/}}

\definecolor{linkcolor}{HTML}{000000}
\definecolor{urlcolor}{HTML}{0085FF}
\hypersetup{pdfstartview=FitH,
  linkcolor=linkcolor,
  urlcolor=urlcolor,
colorlinks=true}

\DeclarePairedDelimiter{\floor}{\lfloor}{\rfloor}

\newcommand{\plot}[3]{
  \begin{figure}[H]
    \begin{center}
      \includegraphics[scale=0.6]{#1}
      \caption{#2}
      \label{#3}
    \end{center}
  \end{figure}
}

\begin{document}
\begin{titlepage}
  \begin{center}
    {\large Санкт-Петербургский политехнический университет\\Петра Великого\\}
  \end{center}

  \begin{center}
    {\large Физико-механический иститут}
  \end{center}


  \begin{center}
    {\large Кафедра «Прикладная математика»}
  \end{center}

  \vspace{8em}

  \begin{center}
    {\bfseries Отчёт по лабораторной работе №2 \\по дисциплине «Компьютерные сети»}
  \end{center}

  \vspace{5em}

  \begin{flushleft}
    \hspace{16em}Выполнил студент:\\\hspace{16em}Мишутин Дмитрий Валерьевич\\\hspace{16em}группа: 5040102/20201

    \vspace{2em}

    \hspace{16em}Проверил:\\\hspace{16em}к.ф.-м.н., доцент\\\hspace{16em}Баженов Александр Николаевич

  \end{flushleft}


  \vspace{6em}


  \begin{center}
    Санкт-Петербург\\2024 г.
  \end{center}

\end{titlepage}

\newpage

\tableofcontents
\listoffigures
\listoftables
\newpage

\section{Постановка задачи}
\quad Требуется реализовать протокол маршрутизации Open Shortest Path First
(OSPF). Рассмотреть работу протокола для линейной, кольцевой и звездной
топологий.

\quad Необходимо выяснить зависимость времени работы и количество посланных
сообщений от размера плавающего окна и вероятности потери сообщения для каждого
протокола и сравнить их друг с другом.

\section{Теория}
\quad OSPF --- протокол динамической маршрутизации, в основе работы которого
лежит представление множества сетей, маршрутизаторов и каналов в виде
ориентированного графа.

\quad Описание работы протокола:
\begin{enumerate}
  \item после включения маршрутизатора протокол ищет непосредственно
    подключенных соседей и устанавливает с ними связь;
  \item Строится карта сети: между соседями происходит обмен информацией о
    подключенных и доступных сетях. Данная карта одинакова на всех
    маршрутизаторах;
  \item Запускается алгоритм SPF (Shortest Path First), который рассчитывает
    оптимальный маршрут к каждой сети. Процесс представляет собой поиск
    кратчайшего пути в графе, в вершинах которого доступные сети, а ребра ---
    пути между ними.
\end{enumerate}

\section{Реализация}
\quad Из языка Python 3.12.2 были использованы следующие модули:
\begin{itemize}
  \item ``numpy'' --- генерация множества чисел;
  \item ``matplotlib.pyplot'' --- построение и отображение графиков;
  \item ``math'' --- экземпляр числа, равного бесконечности.
\end{itemize}

\section{Результаты}
\quad Количество узлов во всех топологиях равно $5$. Рассмотриваем линейную
топологию с радиусом соединения $5$, кольцевую топологию с радиусом соединения
$6$, звездную топологию с радиусом соединения $5$.

Для сети с линейной топологией имитируем падение одного из некрайних узлов и
перенумируем оставшиеся узлы сети. Для сети с кольцевой топологией Имитируем
падение случайного узла и перенумируем оставшиеся узлы сети. Для сети со
звездной топологией имитируем падение центрального узла.

\plot{./images/TopologyStar.png}{Сеть со звездной топологией}{p:sl}

\begin{table}[H]
  \caption{Список путей в сети со звездной топологией}
  \begin{tabular}{| c | c | c | c | c | c |}
    \hline
             & 0 & 1 & 2 & 3 & 4 \\
             \hline
    0 & 0 & 0-4-1 & 0-4-2 & 0-4-3 & 0-4 \\
    \hline
    1 & 1-4-0 & 1 & 1-4-2 & 1-4-3 & 1-4 \\
    \hline
    2 & 2-4-0 & 2-4-1 & 2 & 2-4-3 & 2-4 \\
    \hline
    3 & 3-4-0 & 3-4-1 & 3-4-2 & 3 & 3-4 \\
    \hline
    4 & 4-0 & 4-1 & 4-2 & 4-3 & 4 \\
    \hline
  \end{tabular}
  \centering
\end{table}

\plot{./images/TopologyStar_corrupted.png}{Поврежденная сеть со звездной
топологией}{p:sl_c}

\begin{table}[H]
  \caption{Список путей в поврежденнной сети со звездной топологией}
  \begin{tabular}{| c | c | c | c | c |}
    \hline
             & 0 & 1 & 2 & 3 \\
             \hline
    0 & 0 & - & - & - \\
    \hline
    1 & - & 1 & - & - \\
    \hline
    2 & - & - & 2 & - \\
    \hline
    3 & - & - & - & 3 \\
    \hline
  \end{tabular}
  \centering
\end{table}

\plot{./images/TopologyRing.png}{Сеть с кольцевой топологией}{p:rl}

\begin{table}[H]
  \caption{Список путей в сети с кольцевой топологией}
  \begin{tabular}{| c | c | c | c | c | c |}
    \hline
             & 0 & 1 & 2 & 3 & 4 \\
             \hline
    0 & 0 & 0-1 & 0-1-2 & 0-4-3 & 0-4 \\
    \hline
    1 & 1-0 & 1 & 1-2 & 1-2-3 & 1-0-4 \\
    \hline
    2 & 2-1-0 & 2-1 & 2 & 2-3 & 2-3-4 \\
    \hline
    3 & 3-4-0 & 3-2-1 & 3-2 & 3 & 3-4 \\
    \hline
    4 & 4-0 & 4-0-1 & 4-3-2 & 4-3 & 4 \\
    \hline
  \end{tabular}
  \centering
\end{table}

\plot{./images/TopologyRing_corrupted.png}{Поврежденная сеть с кольцевой
топологией}{p:rl_c}

\begin{table}[H]
  \caption{Список путей в поврежденнной сети с кольцевой топологией}
  \begin{tabular}{| c | c | c | c | c |}
    \hline
             & 0 & 1 & 2 & 3 \\
             \hline
    0 & 0 & 0-1 & 0-3-2 & 0-3 \\
    \hline
    1 & 1-0 & 1 & 1-0-3-2 & 1-0-3 \\
    \hline
    2 & 2-3-0 & 2-3-0-1 & 2 & 2-3 \\
    \hline
    3 & 3-0 & 3-0-1 & 3-2 & 3 \\
    \hline
  \end{tabular}
  \centering
\end{table}

\plot{./images/TopologyLine.png}{Сеть с линейной топологией}{p:tl}

\begin{table}[H]
  \caption{Список путей в сети с линейной топологией}
  \begin{tabular}{| c | c | c | c | c | c |}
    \hline
             & 0 & 1 & 2 & 3 & 4 \\
             \hline
    0 & 0 & 0-1 & 0-1-2 & 0-1-2-3 & 0-1-2-3-4 \\
    \hline
    1 & 1-0 & 1 & 1-2 & 1-2-3 & 1-2-3-4 \\
    \hline
    2 & 2-1-0 & 2-1 & 2 & 2-3 & 2-3-4 \\
    \hline
    3 & 3-2-1-0 & 3-2-1 & 3-2 & 3 & 3-4 \\
    \hline
    4 & 4-3-2-1-0 & 4-3-2-1 & 4-3-2 & 4-3 & 4\\
    \hline
  \end{tabular}
  \centering
\end{table}

\plot{./images/TopologyLine_corrupted.png}{Поврежденная сеть с линейной
топологией}{p:tl_c}

\begin{table}[H]
  \caption{Список путей в поврежденнной сети с линейной топологией}
  \begin{tabular}{| c | c | c | c | c |}
    \hline
             & 0 & 1 & 2 & 3 \\
             \hline
    0 & 0 & 0-1 & - & - \\
    \hline
    1 & 1-0 & 1 & - & - \\
    \hline
    2 & - & - & 2 & 2-3 \\
    \hline
    3 & - & - & 3-2 & 3 \\
    \hline
  \end{tabular}
  \centering
\end{table}

\section{Выводы}
\quad Из полученных результатов можно заметить следующее. Сеть с линейной
топологией наиболее чувствительна к потерям узлов сети, потеря одного узла ведёт
к появлению недостижимых узлов. Сеть с кольцевидной топологией менее
чувствительна к потерям узлов, при потере одного узла она переходит в сеть с
линейной топологией. Сеть со звёздной топологией наименее чувствительна к потере
узлов до тех пор, пока это не центральный узел. В случае потери центрального
узла любая пара других узлов становится недостижима.

\section{Литература}
\begin{itemize}
  \item \href{https://elib.spbstu.ru/dl/2/s20-76.pdf/info}{Баженов А.Н.
    <<Интервальный анализ. Основы теории и учебные примеры: учебное пособие>>};
  \item \href{https://elib.spbstu.ru/dl/5/tr/2021/tr21-169.pdf/info}{Баженов
      А.Н. <<Естественнонаучные и технические применения интервального анализа:
    учебное пособие>>};
  \item \href{https://github.com/AlexanderBazhenov/ComputerNetworks}{Баженов
    А.Н. Репозиторий ``ComputerNetworks'' на GitHub};
\end{itemize}

\section{Приложения}
\quad Исходники лабораторной работы выложены на
\href{https://github.com/MeShootIn/computer-networks/tree/lab-2}{GitHub}.

\end{document}
