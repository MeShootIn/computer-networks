\documentclass[a4paper,12pt]{article}

\usepackage[hidelinks]{hyperref}
\usepackage{amsmath}
\usepackage{mathtools}
\usepackage{shorttoc}
\usepackage{cmap}
\usepackage[T2A]{fontenc}
\usepackage[utf8]{inputenc}
\usepackage[english, russian]{babel}
\usepackage{xcolor}
\usepackage{graphicx}
\usepackage{float}
\graphicspath{{./images/}}

\definecolor{linkcolor}{HTML}{000000}
\definecolor{urlcolor}{HTML}{0085FF}
\hypersetup{pdfstartview=FitH,
  linkcolor=linkcolor,
  urlcolor=urlcolor,
colorlinks=true}

\newcommand{\plot}[3]{
  \begin{figure}[H]
    \begin{center}
      \includegraphics[scale=0.6]{#1}
      \caption{#2}
      \label{#3}
    \end{center}
  \end{figure}
}

\newcommand{\GoBackN}{``Go-Back-N''}
\newcommand{\SelectiveRepeat}{``Selective Repeat''}

\begin{document}
\include{title}
\newpage

\tableofcontents
\listoffigures
\newpage

\section{Постановка задачи}
\quad Требуется реализовать систему из 2-х объектов --- отправителя (Sender) и
получателя (Receiver) ---, в которой участники будут обмениваться сообщениями по
каналу связи с помощью протоколов автоматического запроса повторной передачи со
скользящего окном: {\GoBackN} и {\SelectiveRepeat}.

\quad Необходимо выяснить зависимость времени работы и количество посланных
сообщений от размера плавающего окна и вероятности потери сообщения для каждого
протокола и сравнить их друг с другом.

\section{Теория}
\quad Протоколы {\GoBackN} и {\SelectiveRepeat} являются протоколами скользящего
окна: доставка сообщений происходит в рамках некоторого окна фиксированного
размера. Ошибки выявляются и исправляются в рамках окна.

\quad Основное различие между этими 2-мя протоколами в том, что после
обнаружения подозрительного или поврежденного сообщения протокол {\GoBackN}
повторно передает все сообщения, не получившие подтверждения о получении, тогда
как протокол {\SelectiveRepeat} повторно передает только то сообщение, которое
оказалось повреждено.

\section{Реализация}
\quad Из языка Python 3.12.2 были использованы следующие модули:
\begin{itemize}
  \item ``numpy'' --- генерация множества чисел;
  \item ``matplotlib.pyplot'' --- построение и отображение графиков;
  \item ``time'' --- замерка времени выполнения;
  \item ``enum'' --- создание типа с ограниченным множеством значений;
  \item ``Thread'' --- многопоточность.
\end{itemize}

\section{Результаты}
\quad Оценка эффективности использования протоколов производится по числу
сообщений, которые пришлось отправить, и по времени работы, необходимому для
получения всех сообщений без ошибок. Рассматриваются зависимости этих метрик от
ширины окна и вероятности потери сообщения.

\quad Рассмотрим зависимость этих метрик от размера окна и вероятности потери
сообщения.

\quad По умолчанию число сообщений равно 100, ширина окна 15, вероятность потери
сообщения 0.3.

\plot{./images/GBN_SR_num_prob.png}{Зависимость числа сообщений от вероятности
потери сообщения}{p:n_pr}

\plot{./images/GBN_SR_time_prob.png}{Зависимость времени работы от вероятности
потери сообщения}{p:t_pr}

\plot{./images/GBN_SR_num_ws.png}{Зависимость числа сообщений от ширины
окна}{p:n_ws}

\plot{./images/GBN_SR_time_ws.png}{Зависимость времени работы от ширины
окна}{p:t_ws}

\section{Выводы}
\quad По вышеизложенным результатам можно заметить, что в одинаковых условиях
протоколу {\SelectiveRepeat} требуется отправить меньше сообщений, чем протоколу
{\GoBackN}.

\quad Что ожидаемо, в силу разной обработки и повторной передачи потерянных
сообщений. Протокол {\SelectiveRepeat} работает значительно быстрее протокола
{\GoBackN}.

\section{Литература}
\begin{itemize}
  \item \href{https://elib.spbstu.ru/dl/2/s20-76.pdf/info}{Баженов А.Н.
    <<Интервальный анализ. Основы теории и учебные примеры: учебное пособие>>};
  \item \href{https://elib.spbstu.ru/dl/5/tr/2021/tr21-169.pdf/info}{Баженов
    А.Н. <<Естественнонаучные и технические применения интервального анализа:
  учебное пособие>>};
  \item \href{https://github.com/AlexanderBazhenov/ComputerNetworks}{Баженов
    А.Н. Репозиторий ``ComputerNetworks'' на GitHub};
\end{itemize}

\section{Приложения}
\quad Исходники лабораторной работы выложены на
\href{https://github.com/MeShootIn/computer-networks/tree/lab-1}{GitHub}.

\end{document}
